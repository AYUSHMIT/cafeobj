\documentclass[landscape]{slides}

\topmargin=-2cm
\textheight=6.5in
\input{epsf}
\pagestyle{empty}

\def\DEF{\mathrel{\stackrel{_\triangle}{=}}}

\begin{document}

\begin{slide}\Large
  \begin{center}
    \textbf{CafeOBJ as a Tool for Software Model Checking}

    \bigskip\bigskip\bigskip
    
    Akira Mori (JAIST, Japan)\\
    Toshimi Sawada (SRA, Japan)\\
    Kokichi Futatsugi (JAIST, Japan)
  \end{center}
\end{slide}

\begin{slide}\large\parskip=0pt
  \textbf{System Description}

  \begin{itemize}\itemsep=0pt
  \item automatic safety model checker for (\emph{infinite})
    abstract state machines (ASMs)
  \item ASM defined as special algebra (hidden algebra)
    \begin{itemize}
    \item behavioral specification
    \item input and output as abstract data types (ADTs)
    \item supported in \textbf{CafeOBJ} system
    \end{itemize}
  \item model checking conducted using predicate calculus
    \begin{itemize}
    \item predicate as set of states
    \item previous states by predicate transformer
    \item \textbf{PigNose}: resolution engine for CafeOBJ
    \end{itemize}
  \item \emph{same logic} used for specification and verification
  \end{itemize}
  
\end{slide}

\begin{slide}\large\parskip=0pt
  \textbf{Brief Overview of CafeOBJ}\bigskip

  \begin{itemize}\itemsep=0pt
  \item algebraic specification in tradition of \textsf{OBJ3}\\
    {\normalsize abstract data types, order-sorted algebra\slash
      equational logic, parameterized modules, module expressions,
      term rewriting}\bigskip
  \item new features
    \begin{itemize}
    \item behavioral specification based on hidden algebra
    \item predicate calculus and resolution engine (\textsf{PigNose})
    \item safety model checking and behavioral coinduction
    \end{itemize}
  \end{itemize}
\end{slide}

\begin{slide}\large\parskip=0pt
\textbf{Algebraic Spec. of Dynamic System Behaviors}

\bigskip Examples: Java Bank Account Object\bigskip
\begin{verbatim}public class Account {
  public int balance = 0;
  public void deposit(int amount) {
    if (0 <= amount) balance += amount;
  }
  public void withdraw(int amount) {
    if (amount <= balance) balance -= amount;
  }
}
\end{verbatim}
\end{slide}

\begin{slide}\large\parskip=0pt\baselineskip=.65\baselineskip
\begin{verbatim}
mod* ACCOUNT {
protecting(INT)                        -- data type
*[ Account ]*                          -- hidden sort
op new-account : -> Account            -- new account
bop balance : Account -> Int           -- attribute
bop deposit : Int Account -> Account   -- method
bop withdraw : Int Account -> Account  -- method
var A : Account    vars N : Int
ax balance(new-account) = 0 .
ax 0 <= N -> balance(deposit(N,A)) = balance(A) + N .
ax ~(0 <= N) -> balance(deposit(N,A)) = balance(A) .
ax N <= balance(A) ->
            balance(withdraw(N,A)) = balance(A) - N .
ax ~(N <= balance(A)) ->
            balance(withdraw(N,A)) = balance(A) .
}
\end{verbatim}
\end{slide}

\begin{slide}\large\parskip=0pt
\textbf{Behavioral Spec. based on Hidden Alg.}

\bigskip

\begin{itemize}\itemsep=10pt
\item abstract data type + abstract state machine
\item hidden sort (states) vs. visible sort (data types)
\item only one hidden sort in co-arity of behavioral operation
  (methods and attributes)
\item covers well object-oriented concepts
\end{itemize}
\end{slide}

\begin{slide}\large\parskip=0pt
\textbf{Hidden Algebra as State Machine}

\vspace*{-1.5cm}
\hfil\epsfxsize=.9\textwidth\epsfbox{asm.eps}\hfil\\
\textbf{Model checking?}
\begin{itemize}\itemsep=0pt
\item transitions parameterized 
\item state space unbounded
\item must combine deduction and exploration
\end{itemize}
\end{slide}

\begin{slide}\large\parskip=0pt
\textbf{Invariant Checking for Bank Account}

\bigskip prototype of safety model checking

\bigskip\bigskip

$\mathtt{balance(A)\geq 0}$ for any reachable state \texttt{A} of
\texttt{ACCOUNT}

\bigskip

\begin{itemize}\itemsep=0pt
\item $\mathtt{balance(new\texttt{-}account)\geq 0}$
\item $\mathtt{\forall[A:Account,N:Int]}$\\
  $\mathtt{balance(A)\geq 0\Rightarrow balance(deposit(N,A))\geq 0}$\\
  $\mathtt{balance(A)\geq 0\Rightarrow balance(withdraw(N,A))\geq 0}$
\end{itemize}
\end{slide}

\begin{slide}\large\parskip=0pt\baselineskip=.6\baselineskip
\textbf{State Identification in Hidden Algebra}

\bigskip

\begin{verbatim}
mod* FLAG {
  *[ Flag ]*
  bops (up_) (dn_) (rev_) : Flag -> Flag
  bop up?_ : Flag -> Bool
  var F : Flag
  eq up? up F = true .
  eq up? dn F = false .
  eq up? rev F = not up? F .
 }
\end{verbatim}

\hfil\epsfysize=.35\textheight\epsfbox{flag.eps}\hfil

\bigskip

\hfil$\mathtt{\forall[F:Flag]\;rev\; rev\; F = F}$?\hfil

\end{slide}

\begin{slide}\large\parskip=0pt
  \textbf{Behavioral Abstraction of States}\\
  Hidden elements equivalent iff behaviors (results of observation via
  methods and attributes) are the same.

\bigskip

\textbf{Coinduction as Relational Invariant}

$\mathtt{\forall[F:Flag]\; rev(rev(F))=F}$
\begin{itemize}\itemsep=0pt
\item $\mathtt{\forall[F:Flag]\; up?(rev(rev(F)))=up?(F)}$
\item $\mathtt{\forall[F,F':Flag]}$\\
  $\mathtt{up?(F)=up?(F')\Rightarrow up?(up(F))=up?(up(F'))}$\\
  $\mathtt{up?(F)=up?(F')\Rightarrow up?(dn(F))=up?(dn(F'))}$\\
  $\mathtt{up?(F)=up?(F')\Rightarrow up?(rev(F))=up?(rev(F'))}$
\item relation $\mathtt{up?(\_)=up?(\_)}$ is invariant starting from
  $\mathtt{(rev(rev(F)),F)}$
\end{itemize}

\end{slide}

\begin{slide}\large\parskip=0pt
  \textbf{Symbolic Manipulation of ASM in Hidden Algebra}
  \begin{itemize}\itemsep=0pt
  \item Predicate as set of states\\
    $\mathtt{P(X:Protocol)\DEF}$\\
    $\mathtt{\forall[I,J:Nat]\;flag(I,X)=flag(J,X)=shared\Rightarrow}$\\
    \hspace*{48mm}$\mathtt{cdata(I,X)=cdata(J,X).}$\\
    (must be defined in terms of attributes!)
  \item Predicate transformer as previous state function
    \begin{displaymath}
      \mathtt{pre(P(X:h)_h)_{h'}\DEF
        \sum_{\sigma:wh'\rightarrow h}\exists[V:w]\;P(\sigma(V,Y:h'))_h}
    \end{displaymath}
  \item $\mathtt{Q(X)\DEF\exists[I:Index,M:Data]\;\neg P(write(I,M,X))}$\\
    set of states whose next states via \texttt{write} operation may
    not satisfy cache coherence
  \end{itemize}
\end{slide}


\begin{slide}\large\parskip=0pt
  \textbf{Backward Safety Model Checking}

  \bigskip\bigskip

  To show that $I\not\subseteq\mathtt{pre}^*(\neg P)$

  ($I$: initial states, $P$: safety predicate)

  \bigskip\bigskip
  
  $\mathtt{pre}^*(P)\DEF\mu Z.P\vee\mathtt{pre}(Z)$\\
  (least fixpoint of $\mathtt{pre}$ including $P$)\\
  set of all states from which a state satisfying $P$ may be reached

  \bigskip\bigskip

  Counterexample when $I\subseteq\mathtt{pre}^*(\neg P)$

  \bigskip\bigskip

  Fixpoints can be calculated by predicated calculus!!

\end{slide}  

\begin{slide}\large\parskip=0pt
\textbf{Invariant Checking as Safety Model Checking}

\bigskip

safety predicate: $\mathtt{P(X:Account)\DEF balance(A)\geq 0}$

\bigskip\bigskip

\hfil\epsfxsize=.7\textwidth\epsfbox{pre.eps}\hfil
\normalsize
\begin{center}
$\mathtt{(\exists[N:Int]\;\neg P(deposit(N,A)))\;\vee}$\\
$\mathtt{(\exists[N:Int]\;\neg P(withdraw(N,A)))\Rightarrow \neg P(A)}$\\
$\Updownarrow$\\
$\mathtt{P(A)\Rightarrow (\forall[N:Int]\;P(deposit(N,A))\;\wedge}$\\
$\mathtt{(\forall[N:Int]\;P(deposit(N,A))}$\\
\end{center}


\end{slide}


\begin{slide}\large\parskip=0pt
\textbf{Resolution Engine PigNose}
\begin{itemize}
\item refutation engine for CafeOBJ
\item first-order predicate logic with equality
\item iterative calculation of least fixpoints of $\mathtt{pre}$
\item many-sorted (hyper-)resolution \slash paramodulation
\item SOS (Set of Support) strategy, demodulation
\item formula (meta-)rewriting
\item useful for detecting bugs (attribute clash)
\end{itemize}
\end{slide}

\begin{slide}\large\parskip=0pt
  \vfil
  \begin{center}
    \framebox{\LARGE PigNose Demo}

    \bigskip

    invariant for bank account
  \end{center}
  \vfil
\end{slide}

\begin{slide}\large\parskip=0pt\baselineskip=.6\baselineskip
\textbf{2-process Bakery Algorithm}

\bigskip\bigskip

\begin{verbatim}
N1, N2: Integer := 0;
task body P1 is
begin
  loop
a1:   Non_Critical_Section_1;
b1:   N1 := N2 + 1;
c1:   loop exit when N2 = 0 or N1 <= N2; end loop;
d1:   Critical_Section_1;
e1:   N1 := 0;
  end loop;
end P1;

task body P2 is
...
c2:   loop exit when N1 = 0 or N2 < N1; end loop;
...
\end{verbatim}
\end{slide}

\begin{slide}\large\parskip=0pt\baselineskip=.53\baselineskip
\vspace*{-2cm}
\begin{verbatim}
mod! NATNUM {
  protecting(FOPL-CLAUSE)
  [ NatNum ]
  op 0 : -> NatNum
  op s : NatNum -> NatNum
  pred _<_ : NatNum NatNum  -- { meta-demod }
  vars M N : NatNum
  ax ~(s(M) < M) .         ax ~(s(M) = 0) .
  ax [SOS]: M < s(M) .     ax [SOS]: 0 < s(M) .
  ax ~(s(M) = M) .         ax [SOS]: M = 0 | 0 < M .
  ax ~(0 < M)| ~(M = 0) .  ax ~(M = N & M < N) .
  ax ~(M < N & N < M) .    ax ~(M < 0) .
  ax M = M .
}

mod! STATUS {
  protecting(FOPL-CLAUSE)
  [ Status ]
  ops non-CS wait CS : -> Status
  var S : Status
  ax (S = S) = true .
}
\end{verbatim}
\end{slide}

\begin{slide}\large\parskip=0pt\baselineskip=.53\baselineskip
\vspace*{-2cm}
\begin{verbatim}
mod* CUSTOMER1 {
protecting(NATNUM + STATUS)
*[ Customer1 ]*
op init1 : -> Customer1
bop ticket1 : Customer1 -> NatNum
bop stat1 : Customer1 -> Status
bop run1 : Customer1 NatNum -> Customer1
vars C C1 : Customer1  vars M N : NatNum
eq stat1(init1) = non-CS .   eq ticket1(init1) = 0 .
ax stat1(C) = non-CS -> stat1(run1(C,M))= wait .
ax stat1(run1(C,M))= wait ->
           stat1(C) = non-CS | stat1(C) = wait .
ax stat1(C) = non-CS -> ticket1(run1(C,M)) = s(M) .
ax stat1(C) = wait & (M = 0 | ~(M < ticket1(C))) ->
                             stat1(run1(C,M)) = CS .
ax stat1(run1(C,M)) = CS -> stat1(C) = wait .
ax stat1(C) = wait & ~(M = 0) & M < ticket1(C) ->
                           stat1(run1(C,M)) = wait .
ax stat1(C) = wait -> ticket1(run1(C,M)) = ticket1(C) .
ax (stat1(C) = CS) = (stat1(run1(C,M)) = non-CS) .
ax stat1(C) = CS -> ticket1(run1(C,M)) = 0 .
}
\end{verbatim}
\end{slide}

\begin{slide}\large\parskip=0pt\baselineskip=.53\baselineskip
\vspace*{-2cm}
\begin{verbatim}
mod* CUSTOMER2 {
protecting(NATNUM + STATUS)
*[ Customer2 ]*
op init2 : -> Customer2
bop ticket2 : Customer2 -> NatNum
bop stat2 : Customer2 -> Status
bop run2 : Customer2 NatNum -> Customer2
vars C C1 : Customer2  var M : NatNum
eq stat2(init2) = non-CS .   eq ticket2(init2) = 0 .
ax stat2(C) = non-CS -> stat2(run2(C,M))= wait .
ax stat2(run2(C,M))= wait ->
                  stat2(C) = non-CS | stat2(C) = wait .
ax stat2(C) = non-CS -> ticket2(run2(C,M)) = s(M) .
ax stat2(C) = wait & (M = 0 | ticket2(C) < M) ->
                                stat2(run2(C,M)) = CS .
ax stat2(run2(C,M)) = CS -> stat2(C) = wait .
ax stat2(C) = wait & ~(M = 0) & ~(ticket2(C) < M) ->
                             stat2(run2(C,M)) = wait .
ax stat2(C) = wait -> ticket2 (run2(C,M)) = ticket2(C) .
ax (stat2(C) = CS) = (stat2(run2(C,M)) = non-CS) .
ax stat2(C) = CS -> ticket2(run2(C,M)) = 0 .
}
\end{verbatim}
\end{slide}

\begin{slide}\large\parskip=0pt\baselineskip=.65\baselineskip
\vspace*{-1.5cm}
\begin{verbatim}
mod* SHOP {
protecting(CUSTOMER1 + CUSTOMER2)
*[ Shop ]*
op Init : -> Shop
op shop : Customer1 Customer2 -> Shop { coherent }
bops Run1 Run2 : Shop -> Shop
bops Stat1 Stat2 : Shop -> Status
bops Ticket1 Ticket2 : Shop -> NatNum
var C1 : Customer1  vars C2 : Customer2  var S : Shop
eq Init = shop(init1,init2) .
ax Run1(shop(C1,C2)) = shop(run1(C1,ticket2(C2)),C2) .
ax Run2(shop(C1,C2)) = shop(C1,run2(C2,ticket1(C1))) .
eq Stat1(shop(C1,C2)) = stat1(C1) .
eq Stat2(shop(C1,C2)) = stat2(C2) .
eq Ticket1(shop(C1,C2)) = ticket1(C1) .
eq Ticket2(shop(C1,C2)) = ticket2(C2) .
}
\end{verbatim}
\end{slide}

\begin{slide}\large\parskip=0pt
  \textbf{Coherent Operation as ASM Constructor}
  \begin{itemize}\itemsep=0pt
  \item[\bf --] \texttt{shop : Customer1 Customer2 -> Shop \{ coherent
      \}}
  \item[\bf --] more than one hidden arguments
  \item[\bf --] equational reasoning (congruence rule) not sound
  \item[\bf --] \textbf{coherent} if preserves behavioral equivalence\\
    $\rightarrow$ equational reasoning sound
  \item[\bf --] used to specifying system configuration in
    \emph{reachable states} $\rightarrow$ search space drastically
    reduced
  \item[\bf --] coherence must be proved separately (easy for \texttt{shop})\\
    (syntactic criteria of \textbf{[Bidoit,Hennicker]})
  \end{itemize}
  
\end{slide}

\begin{slide}\large\parskip=0pt
  \textbf{Safety Model Checking of Bakery Algorithm}

  \bigskip\bigskip

  mutual exclusion:\\
  $\mathtt{P(S:Shop)\DEF\neg(Stat1(S) = CS\wedge Stat2(S) = CS)}$

  \bigskip\bigskip

  \begin{itemize}\itemsep=0pt
  \item[\bf --] difficult to automate
  \item[\bf --] ticket number unbounded
  \item[\bf --] finite model checking requires human abstraction
  \item[\bf --] fixpoint not obtained in one step
  \end{itemize}
\end{slide}

\begin{slide}\large\parskip=0pt
  \vfil
  \begin{center}
    \framebox{\LARGE PigNose Demo}

    \bigskip

    interleaving mutual exclusion

    \bigskip

    error detection with simultaneous entry
  \end{center}
  \vfil
\end{slide}

\begin{slide}\large\parskip=0pt
  \textbf{Results}

  \bigskip\bigskip

  \begin{itemize}\itemsep=0pt
  \item reaches fixpoint in 4th iteration
  \item 23 out of 31 nodes pruned by deduction
  \item takes less than 20 secs on PentiumIII 750MHz
  \item can detect errors when tie-breaker for simultaneous entry is
    not performed
  \item seamlessly integrated, no human intervention, no syntactic
    conversion, no temporal logic formula
  \end{itemize}
\end{slide}

\begin{slide}\large\parskip=0pt

\hspace*{-2cm}\epsfysize=1.3\textheight\epsfbox{bakery.eps}

\end{slide}

\begin{slide}\large\parskip=0pt
  \textbf{Other Examples}

  \begin{itemize}\itemsep=0pt
  \item \texttt{FLAG} coinduction (backward checking)
  \item \texttt{STREAM} coinduction as forward safety checking\\
    (aka. {\normalsize\textbf{circular coinductive rewriting
        [Goguen,Rosu]}})
  \item \texttt{STACK} refinement as pointer and \texttt{ARRAY}
  \item cache coherence for Illinois protocol
  \item alternating bit protocol
  \item Needham-Schroeder public key protocol
  \end{itemize}

  \bigskip\bigskip

  Can handle logically complex system specifications
\end{slide}

\begin{slide}\large\parskip=0pt
  \textbf{Future Work}

  \begin{itemize}\itemsep=0pt
  \item[\bf --] combination with forward checking
  \item[\bf --] domain specific decision procedures\\ (e.g.,
    Presburger Arithmetic)
  \item[\bf --] liveness model checking
  \item[\bf --] fairness via partiality with order sorts
  \item[\bf --] abstract interpretation\\ (widening and narrowing
    \textbf{[Cousot]})
  \item[\bf --] real-time hybrid systems
  \item[\bf --] comparison with other systems: STeP and TLA
  \end{itemize}

\end{slide}

\end{document}
